\documentclass{article}
\usepackage[utf8]{inputenc} %tildes y caracteres especiales
\usepackage[spanish]{babel} 
\usepackage{amsmath, amssymb}
\usepackage{geometry}
\usepackage{fancyhdr}
\usepackage{enumitem}
\usepackage{titlesec}
\usepackage{graphicx}
\usepackage{physics}
\usepackage{xcolor}
\usepackage{ragged2e}

\usepackage{tikz}
\usepackage{pgfplots}
\pgfplotsset{compat=1.18}
\usepackage{tikz-3dplot} % Permite coordenadas 3D

\setlength{\parindent}{0pt}

\geometry{
  top=2.54cm,
  bottom=2.54cm,
  left=2.54cm,
  right=2.54cm
} %MARGENES APA

% Estilo personalizado
\pagestyle{fancy}
\fancyhf{}
\rhead{Taller No. 3-Teoría Electromagnética}
\rfoot{ \thepage}
\renewcommand{\footrulewidth}{0.4pt} % línea sobre pie de página (default: 0pt → invisible)

\newcommand{\problema}[2]{%
  \vspace{0.5cm}
  {\noindent\textbf{Problema #1} #2} 
  \noindent 
}

% Redefinir sección: texto normal, sin espacio extra
\titleformat{\section}
  {\normalfont}       % Estilo del título (normal, sin negrita)
  {\thesection}       % Numeración (ej. 1, 2, 3...)
  {1em}               % Espacio entre número y título
  {}                  % Código antes del título (vacío)


\begin{document}

\begin{figure}[h]
    \centering
    \tdplotsetmaincoords{70}{120} % ángulo de vista

    \begin{tikzpicture}[tdplot_main_coords, scale=3]

        % Ejes coordenados
        \draw[->] (0,0,0) -- (1.5,0,0) node[anchor=north east]{$x$};
        \draw[->] (0,0,0) -- (0,1.5,0) node[anchor=north west]{$y$};
        \draw[->] (0,0,0) -- (0,0,1)   node[anchor=south]{$z$};

        % Coordenadas de trayectoria
        \coordinate (a) at (1,0,0);
        \coordinate (b) at (1,1,0);
        \coordinate (c) at (1,1,1); % Este es el punto final

        % Trayectoria en pasos
        \draw[thick, red, ->] (0,0,0) -- (a) node[anchor=south east] {$(x_0)$} node[midway , above] {$\lambda_1$};
        \draw[thick, blue, ->] (a) -- (b) node[anchor=south west] {$(y_0)$} node[midway , above] {$\lambda_2$};
        \draw[thick, green, ->] (b) -- (c) node[anchor=north west] {$(z_0)$} node   [midway , right] {$\lambda_3$};

        % Punto final
        \filldraw[black] (c) circle (0.5pt);
        \node[anchor=south] at (c) {$(x_0, y_0, z_0)$};

    \end{tikzpicture}
    \caption*{Trayectoria $(x_0, y_0, z_0)$}
\end{figure}

\end{document}