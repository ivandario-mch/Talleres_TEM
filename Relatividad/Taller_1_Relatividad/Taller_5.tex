\documentclass[12pt]{article}
\usepackage[utf8]{inputenc}
\usepackage{amsmath,amssymb}
\usepackage{float}
\usepackage{graphicx}
\usepackage[a4paper, margin=2.54cm]{geometry}
\usepackage{enumerate}
\usepackage{xcolor}
\usepackage{caption}
\usepackage{fancyhdr}
\usepackage{pgfplots}
\usepackage{enumitem}
\usepackage{tikz}
\usepackage{physics}
\usepackage{tikz-3dplot} % Permite coordenadas 3D
% Definir un comando para el número de pregunta en azul
\newcommand{\question}[1]{\textcolor{blue}{\textbf{#1}}}
%\usepackage[spanish]{babel}
\setlength{\headheight}{14.5pt} % Ajustar la altura del encabezado
\usetikzlibrary{3d}
\pagestyle{fancy}
\fancyhf{}
\fancyhead[L]{Taller No. 1- Introducción a la Teoria de la Relatividad}
\fancyfoot[R]{ \thepage}
\renewcommand{\footrulewidth}{0.4pt}



\begin{document}
\begin{titlepage}
        \begin{center}
            \LARGE \textbf{Taller No. 3\\Teoría Electromagnética}
            \vfill
            \large

            Karen Alejandra Freire Rosero\\
            Sonier Andrés Ortiz Castelblanco\\
            Sarah Isabel Tejada García\\
            Santiago Alejandro Pérez Ramos
            \vfill
            \textbf{Asignatura:} Teoría Electromagnética\\
            \textbf{Profesor:} Servio Tulio Pérez Merchancano, Ph.D\\
            \vfill
            Universidad del Cauca\\
            Facultad de Ciencias Naturales, Exactas y de la Educación\\
            Departamento de Física\\
            Popayán, Cauca\\
            2025
        \end{center}
\end{titlepage}

% ----------------------------------------------------------------------

\section*{\question{Problema 1.1}}



\subsection*{Solución}

% ----------------------------------------------------------------------

\section*{\textcolor{blue}{Problema 1.2}}


\subsection*{Solución}

% ----------------------------------------------------------------------

\section*{\textcolor{blue}{Problema 1.3}}


\subsection*{Solución}
% ----------------------------------------------------------------------

\section*{\textcolor{blue}{Problema 1.4}}


\subsection*{Solución}

% ----------------------------------------------------------------------

\section*{\textcolor{blue}{Problema 1.5}}


\subsection*{Solución}

\end{document}