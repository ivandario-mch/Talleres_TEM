\documentclass[12pt]{article}
\usepackage[utf8]{inputenc}
\usepackage{amsmath,amssymb}

\begin{document}

\section*{Identidades de Cálculo Vectorial}

\subsection*{(1) $\nabla(g^2)$}
\[
\nabla(g^2) 
= \left( \frac{\partial}{\partial x}(g^2), \; \frac{\partial}{\partial y}(g^2), \; \frac{\partial}{\partial z}(g^2) \right).
\]
Usando la regla de la derivada,
\[
\frac{\partial}{\partial x}(g^2) 
= 2\,g \,\frac{\partial g}{\partial x}, 
\quad
\frac{\partial}{\partial y}(g^2) 
= 2\,g \,\frac{\partial g}{\partial y}, 
\quad
\frac{\partial}{\partial z}(g^2) 
= 2\,g \,\frac{\partial g}{\partial z}.
\]
Por lo tanto,
\[
\nabla(g^2) 
= 2\,g \,\nabla g.
\]
\[
\text{QED.}
\]

\subsection*{(2a) $\nabla \cdot (\lambda\,\mathbf{A})$}
Sea $\lambda$ una función escalar y $\mathbf{A} = (A_x, A_y, A_z)$ un campo vectorial. Entonces,
\[
\nabla \cdot (\lambda\,\mathbf{A})
= \frac{\partial}{\partial x}(\lambda A_x)
+ \frac{\partial}{\partial y}(\lambda A_y)
+ \frac{\partial}{\partial z}(\lambda A_z).
\]
Aplicando la regla del producto en cada término:
\[
\frac{\partial}{\partial x}(\lambda A_x)
= \frac{\partial \lambda}{\partial x}\,A_x
+ \lambda \,\frac{\partial A_x}{\partial x},
\]
\[
\frac{\partial}{\partial y}(\lambda A_y)
= \frac{\partial \lambda}{\partial y}\,A_y
+ \lambda \,\frac{\partial A_y}{\partial y},
\]
\[
\frac{\partial}{\partial z}(\lambda A_z)
= \frac{\partial \lambda}{\partial z}\,A_z
+ \lambda \,\frac{\partial A_z}{\partial z}.
\]
Sumando,
\[
\nabla \cdot (\lambda\,\mathbf{A})
= \lambda\,(\nabla \cdot \mathbf{A})
+ \mathbf{A} \cdot (\nabla \lambda).
\]
\[
\text{QED.}
\]

\subsection*{(2b) $\nabla \times (\lambda\,\mathbf{A})$}
Para el rotacional de un producto escalar con un vector:
\[
\nabla \times (\lambda\,\mathbf{A})
= \begin{vmatrix}
\mathbf{i} & \mathbf{j} & \mathbf{k} \\[6pt]
\displaystyle \frac{\partial}{\partial x} & \displaystyle \frac{\partial}{\partial y} & \displaystyle \frac{\partial}{\partial z} \\[6pt]
\lambda A_x & \lambda A_y & \lambda A_z
\end{vmatrix}.
\]
Nuevamente, al desarrollar se ve que cada término produce una parte con $\nabla \lambda \times \mathbf{A}$ y otra con $\lambda\,(\nabla \times \mathbf{A})$, por lo que:
\[
\nabla \times (\lambda\,\mathbf{A})
= \nabla \lambda \times \mathbf{A}
+ \lambda \,\bigl(\nabla \times \mathbf{A}\bigr).
\]
\[
\text{QED.}
\]

\subsection*{(3) $\nabla \cdot (\mathbf{A} \times \mathbf{B})$}
Si $\mathbf{A} = (A_x,A_y,A_z)$ y $\mathbf{B} = (B_x,B_y,B_z)$, entonces:
\[
\nabla \cdot (\mathbf{A} \times \mathbf{B})
= \mathbf{B}\cdot(\nabla \times \mathbf{A})
- \mathbf{A}\cdot(\nabla \times \mathbf{B}).
\]
Esta es la conocida identidad que relaciona la divergencia de un producto cruzado con rotacionales:
\[
\text{QED.}
\]

\subsection*{(4) $\nabla \times (\mathbf{A} \times \mathbf{B})$}
La identidad estándar para el rotacional de un producto cruzado es:
\[
\nabla \times (\mathbf{A} \times \mathbf{B})
= (\mathbf{B} \cdot \nabla)\,\mathbf{A}
- (\mathbf{A} \cdot \nabla)\,\mathbf{B}
+ \mathbf{A}\,(\nabla \cdot \mathbf{B})
- \mathbf{B}\,(\nabla \cdot \mathbf{A}).
\]
\[
\text{QED.}
\]

\subsection*{(5) $\nabla \times \bigl[f\,(\mathbf{A} \times \mathbf{B})\bigr]$}
Sea $f$ un escalar y $\mathbf{A}, \mathbf{B}$ campos vectoriales.

\end{document}
